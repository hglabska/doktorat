\section{Laminar Population Analysis (LPA)}
Laminar Population Analysis (LPA) opublikowana w \cite{Einevoll2007} jest jedną z metod analizy danych elektrofizjologicznych. Podobnie jak wiele innych popularnych metod, rozkłada oryginalny sygnał na składowe, które potem próbuje zinterpretować w kontekście aktywności poszczególnych populacji komórek.

Załóżmy, że mamy dany sygnał $\phi(z_i, t_j)$ z $N_{ch}$ elektrod w $N_t$ punktach czasowych. Możemy go przybliżyć sumą kombinacji funkcji zależnych jedynie od czasu i zależnych jedynie od położenia.
\begin{equation}
\phi(z_i, t_j) = \sum_{n}f_n(z_i)g_n(t_j).
\label{product}
\end{equation}

W szeroko stosowanej metodzie analizy głównych składowych (PCA - Prinipal Compomnents Analysis) składowe wybierane są w taki sposób by pierwsza z nich wychwyciła jak największą wriancję sygnału, kolejna jak największą wariancję w pozostałej części sygnału itp. Przy czym składowe te są względem siebie ortogonalne. Metoda skladowych niezależnych (ICA - Independent Components Analysis) wykrywa komponenty statystycznie niezależne. Celem LPA jest wyodrębnienie składowych wywołanych aktywnością poszczególnych populacji. 

Metoda LPA wychodzi z następujących założeń:
\begin{itemize}
\item sygnał LFP jest wywołany przez iglice potencjał czynnościowego, który możemy zaobserwować w sygnale MUA
\item LFP i MUA mają postać produktową, czyli można je zapisać w postaci \ref{product} (gLPA - generalized Laminar Population Analysis dopuszcza pewne osłabienie tego warunku).
\end{itemize}

Analiza LPA składa się dwóch etapów. W pierwszym na podstawie MUA określa się tempa wystrzeliwania iglic potencjału czynnościowego oraz profil przestrzenny MUA. W drugim kroku wyznaczamy wejście synaptyczne z poszczególnych populacji i profilu przestrzennego LFP.

\subsection{Moldelowanie populacyjnego MUA}
Pierwszy etap metody LPA polega na oszacowaniu tempa wystrzeliwania iglic potencjału czynnościowego. Sygnał MUA jest rozpisany na wkłady pochodzące od poszczególnych populacji:
\begin{equation}
MUA^m(z_i, t_j) = \sum^{N_{pop}}_{n=1} M_n(z_i)r_n(t_j).
\label{MUAdecomposition}
\end{equation}
W tym przypadku $N_{pop}$ to liczba populacji, $r_n(t_j)$ - tempo wystrzeliwania iglic potencjału czynnościowego,  $M_n$ - profil przestrzenny MUA dla n-tej populacji. $M_n$ zależy od własności fizjologicznych neuronów w danej populacji, ich rozmieszczenia oraz przewodności elektrycznej ośrodka. Metoda LPA przyjmuje że $M_n$ jest funkcją trapezoidalną, definiowaną za pomocą trzech parametrów: $z_{0n}, \ a_n, \ b_n$. Przyjmuje wartość 0 dla $z_{j}< z_{0n} - a_n/2 - b$ oraz $z_{j} > z_{0n} + a_n/2 + b$, wartość 1 dla $z_{0n} - a_n/2 < z_j < z_{0n} + a_n/2$, jest liniowa w pozostałych przedziałach (patrz fig. \ref{MUAshape}). Dodatkowo zakłada się że $b_n < 0.1 mm$.

W tym etapie wyznaczmy parametry $z_{0n}, \ a_n, \ b_n$, a tym samym determinujemy tempo wystrzeliwania iglic potencjału czynnościowego dla poszczególnych populacji $r_n$ minimalizując błąd \ref{MUAerror}.

\begin{figure}
\caption{MUA shape profile function}
\label{MUAshape}
\includegraphics{../ryciny/MUAshape.png}
\end{figure}

\begin{equation}
e_M = \sum^{N_{ch}}_{i=1}\sum^{N_t}_{j=1} \left( MUA(z_i, t_j) - MUA^m(z_i, t_j) \right)^2/ \sum^{N_{ch}}_{i=1}\sum^{N_t}_{j=1} MUA(z_i, t_j)^2.
\label{MUAerror}
\end{equation}

\subsection{Modelowanie populacyjneg LFP}
W tym kroku korzystamy z założenia że sygnał LFP jest wywołany przez iglice potencjału czynnościowego, który możemy zaobserwować w sygnale MUA oraz że LFP ma postać produktową. Sygnał LFP rozpisuje się jako:
\begin{equation}
 LFP^m(z_i, t_j) = \sum^{N_{pop}}_{n=1}L_n(z_i)(h*r_n)(t_j),
 \label{LFPdecomposition}
\end{equation}
gdzie symbol $*$ oznacza splot co zapisuje się jako:
\begin{equation}
(h_n*r_n)(t_j) = \sum_{k = -\infty}^{\infty}h(t_k)r_n(t_j - t_k)
\label{splot}
\end{equation}

$L_n$ to profil przestrzenny LFP wywołanego aktywnością populacji $n,$ $h_n$ to jądro która reprezentuje opóźnienie w wystąpieniu i propagacji sygnału LFP wywołanego aktywnością populacji $n:$


\begin{eqnarray}
\label{h_function}
h(t_j) = \frac{1}{\tau}e^{-(t_j - \Delta)/\tau}  &for \ t_j \geq 0\\
h(t_j) = 0 &for \ t_j <0.
\end{eqnarray}

Wyznaczamy $\tau,$ $\Delta,$ $L_n$ minimalizując:
\begin{equation}
e_L = \sum^{N_{ch}}_{i=1}\sum^{N_t}_{j=1} \left( LFP(z_i, t_j) - LFP^m(z_i, t_j) \right)^2/ \sum^{N_{ch}}_{i=1}\sum^{N_t}_{j=1} LFP(z_i, t_j)^2.
\label{LFPerror}
\end{equation}

\subsection{ Uogólninie metody LPA - generalized Laminar Population Analysis (gLPA)}

Założenie, że sygnał LFP można zpisać w postaci \ref{product} nie w pełni oddaje rzeczwistość. Biorąc pod uwagę, że różne procesy związane z pobudzniem postsynaptycznym mają różną skalę czasową oraz zjawisko wewnętrznego filtrowania w drzewach dendrytycznych \cite{Linden2010}.% MUA wpływa na LFP też ... ...% ma na niego wpływ efekt wewnętrzenego filtrownia w dzewach dendrytycznych.
Ugólniona metoda LPA (gLPA - generalized Laminar Population Analysis ) dopuszcza $K$ ($K \geq 1$) profili przestrzennych LFP oraz odpowiadających im opóźnień w propagacji i wystąpieniu sygnału. 
\begin{equation}
 gLFP\textrm{-}K^m(z_i, t_j) =  \sum^{N_{pop}}_{n=1} \sum_{k=1}^K L_n^k(z_i)(h^k*r_n)(t_j),
 \label{gLFPdecomposition}
\end{equation}