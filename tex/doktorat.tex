\documentclass[a4paper,12pt,polish]{book}

\textwidth=161mm
\textheight=247mm
\evensidemargin=0mm 
\oddsidemargin=0mm
\topmargin=-12mm

%\documentclass[a4paper,12pt]{book}
\usepackage[utf8]{inputenc}
\usepackage[polish]{babel}
\usepackage[T1]{fontenc}
\usepackage{polski}
\usepackage{graphicx}
\usepackage{amsmath}
\usepackage{comment}
\usepackage{amsfonts}
\usepackage[font={small}]{caption}
\usepackage{float}
\usepackage{rotating}
\usepackage{color}

\usepackage{rotating}
%\usepackage{jneurosci}
\usepackage{setspace}
\usepackage{subfigure}
\newcommand{\g}[1]{\mathbf{#1}}

\usepackage{natbib}
%\usepackage{geometry}
%\geometry{bindingoffset=1cm}
%\evensidemargin=0mm 
%\oddsidemargin=0mm
\begin{document}
\onehalfspacing
%\doublespacing
\thispagestyle{empty}

\widowpenalty=1000 
\clubpenalty=500
 
\begin{center}
  
   \ \\
 

\includegraphics{../ryciny/nencki.png}
\vfill

   \begin{doublespace}
     \hspace*{-4ex}\begin{minipage}[l]{1.1\textwidth}
       \begin{center}
         {\sc\LARGE  
	Walidacja metod analizy rejestracji elektrofizjologicznych \\
 	przy użyciu danych symulacyjnych
      }\\[1em]
         {\large\sc Helena Głąbska}
       \end{center}

     \end{minipage}
   \end{doublespace}

   \vfill
\
 \vfill
  {%\hfill
  \begin{minipage}[c]{0.75\textwidth}
    \centering
    {\sc praca doktorska wykonana pod kierunkiem} \\
    {\sc dr. hab.\ Daniela Wójcika}\\
  \end{minipage}} \\
  \vfill  

  {\sc Instytut Biologii Doświadczalnej} \\
  {\sc im. Marcelego Nenckiego}\\
  {\sc Polska Akademia Nauk}\\[1em]
  {\sc Warszawa $\ast$ 2015}
\end{center}

\newpage
\thispagestyle{empty}
\ 
\newpage
\thispagestyle{empty}
\vspace*{0.5\textheight}

\hspace*{0.5\textwidth} \begin{minipage}[c]{0.5\textwidth}
  %\centering
  {\sc Podziękowania ...}
\end{minipage}

\ 
\newpage


\ 
\newpage





\section*{Streszczenie} 

\tableofcontents
\chapter{Wstęp}
\section{Modelowania układu nerwowego - modele typu Hodgkina-Huxleya}
\subsection{Model Hodgina-Huxleya}
\subsection{Rozszerzenie modelu Hodgina-Huxleya}
\subsection{Symulatory układu nerwowego}
\subsubsection{Neuron}
\subsubsection{GENESIS}
\subsubsection{NEST}
\subsubsection{Brian}
\subsubsection{STEPS}
\subsubsection{XPPAUT}
\section{Potencjał zewnątrzkomórkowy}
\section{Laminar Population Analysis (LPA)}
\section{Rozszerzenie metody LPA -generalized Laminar Population Analysis (gLPA)}
\section{Cel pracy}
\subsection{Walidacja metody Laminar Population Analysis}
\subsection{??  Walidacje metody Independent Component Analysis}
\subsection{Publikacja bogatego zestawu danych na potrzeby walidacji innych metod analizy danych}


\chapter{Dane}
\section{Model pętli korowo-wzgórzowej Traub at al}
\section{Rozbudowa modelu Traub at al}
\section{Wygenerowane zestawy danych}
\subsection{Stymulacja prądem oscylacyjnym}
\subsection{Odpowiedź sieci na wejście z pojedynczej populacji}
\subsection{Blokada szybkich kanałów jonowych}


\chapter{Wyniki}
\section{Walidacja metody generalized Laminar Population Analysis}
\subsection{Uproszczony przypadek - zastąpienie pierwszej fazy gLPA danymi z PSTH}
\subsection{Oryginala metoda gLPA}
\subsection{Modyfikacja założeń dotyczących profilu przestrzennego MUA}


\chapter{Dyskusja}

\chapter{Podsumowanie wyników i wnioski}




\bibliographystyle{agsm}
\bibliography{doktorat}

\end{document}
